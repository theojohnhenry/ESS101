
\documentclass[11pt]{article}
\usepackage{geometry}
\geometry{a4paper}
\linespread{1.1} % Line spacing


% FIGURES AND FLOATS
\usepackage{graphicx} % Required for including pictures
\usepackage{float} % Allows putting an [H] in \begin{figure} to specify the exact location of the figure
\usepackage{wrapfig} % Allows in-line images such as the example fish picture
\usepackage[font={small,it}]{caption}
\usepackage{subcaption}
\usepackage{epstopdf}
\graphicspath{{images/}}



% MATH
\usepackage{amssymb}
\usepackage{amsmath}
\usepackage{algorithm}
\usepackage[noend]{algpseudocode}

% OTHER
\usepackage[]{mcode}
\usepackage{color}



%%%%%%%%%%%%%%%%%%%%%%%%%%%%%%%%%%%%%%%%%%%%%%%%%%%%%%%%%%%%
\title{Hand in X}
\author{Name1 Surname1 \and Name2 Surname2 }

\begin{document}
\maketitle
\section*{About this template}
This template is can  be a starting point for the reports written for the hand-ins in the course ``Modeling and Simulation", fall 2019. The template contains some examples on how to import pictures, make lists, etc. and some examples on equations. The template does not aim to be complete in this sense and for other details, see a Latex manual.

\emph{Note:} If you haven't use Latex before, there is a multitude of ``getting started" guides on the internet, and several different editors (for instance, LyX or TeXmaker). Starting to use Latex  is usually associated with a somewhat steep learning curve, but it is worth the effort! The result is usually much better looking than ``WYSIWYG"-editors like \emph{Word}, and mathematical expressions in particular is much easier to handle. On top of this, you will most likely write your Thesis in Latex, so there is every reason to get started now! When you start: Are you getting errors? Google is your friend!
\newpage
\section{Introduction}
Write a small introduction to the problem, but keep it concise.
\section{Question a)}
\begin{itemize}
\item Make sure that you supply the answer that is requested
\item Make sure that your line of reasoning is clear  - remember that we need to be able to understand what you have done through the report.
\item Don't forget to define all symbols you introduce.
\end{itemize}
\section{Question b)}
\section{Question c)}
\newpage
\begin{center}
{\huge Useful Examples}
\end{center}
\section{Section}

\subsection{Subsection}
\subsubsection{Sub-Subsection}
\paragraph{Paragraph}

\section{Writing math}
This is how you write an equation
\begin{equation}\label{eq:eq1}
x(k+1) = rx(k)(1-x(k)).
\end{equation}
Alternatively, if you want to write the same equation ``inline", you write it like this $x(k+1) = rx(k)(1-x(k))$.
If you want to refer to the equation you do it like this \eqref{eq:eq1}. The use of ``eq:" in the labeling of equations (and other things) allows you do group your references and access them more easy.
If you want to list several equations in a nice way you can do it like this
\begin{align}
x(k+1) 	     & = x(k) + K \text{cos}(\varphi(k)) \\
\varphi(k+1) & = \varphi(k) + x(k).
\end{align}
If you want to break up a long equation you can do it like this
\begin{align}
\begin{split}
(x+y)^3 	&= (x+y)^2(x+y)\\
&=(x^2+2xy+y^2)(x+y)\\
&=(x^3+2x^2y+xy^2) + (x^2y+2xy^2+y^3)\\
&=x^3+3x^2y+3xy^2+y^3
\end{split}
\end{align}
An optimization problem can be written in the following way
\begin{subequations}\label{eq:subeq}
\begin{align}
\underset{x}{\min}	&\quad f(x) \\
\text{s.t.}				&\quad h(x) \leq 0 \label{eq:con1}\\
								&\quad g(x) = 0 \label{eq:con2}
\end{align}
\end{subequations}
When using subequations you can either refer to all of the equations like this \eqref{eq:subeq}, or to individual sub-equations like this \eqref{eq:con1},\eqref{eq:con2}.

If you want to write matrices you can write
\begin{equation}
\underbrace{
\left[
\begin{array}{cc}
a_{11} & a_{12}\\
a_{21} & a_{22}
\end{array}
\right]
}_{A}
\left[
\begin{array}{c}
x_1 \\
x_2
\end{array}
\right]
=
\left[
\begin{array}{c}
b_1 \\
b_2
\end{array} \right]
\end{equation}
As in the other cases, \& demarcates columns and $\backslash\backslash$ demarcates rows
Larger parenthesis are done in a similar way, e.g.,
\begin{equation}
x(t) = \left\{ \left( \sum_{k=0}^\infty \frac{t^k}{k!}A^k\right)x_0 \right\}
\end{equation}
The following can also be usefull $\leq \  \geq \ \bar{x} \ \hat{x} \ \dot{x} \ \ddot{x} \ \tilde{x} $ as can this
\begin{equation}
\int_{-\infty}^{\infty}f(x)\mathrm{d}x, \ \frac{\partial f(x)}{\partial x}, \ \frac{\mathrm{d}f(x)}{\mathrm{d}x} \ \nabla_x f(x)
\end{equation}
\section{Figures}
A figure can be included like this
\begin{figure}[h]
\begin{center}
\includegraphics[scale=0.7]{stdmap.png}
\caption{ A figure with caption \label{fig:fig1}}
\end{center}
\end{figure}
and can when needed be referenced like this Figure \ref{fig:fig1}.
Figures side by side can be done like this and can be referred to either like this Figure \ref{fig:sub1} and Figure \ref{fig:sub2}, or like this Figure \eqref{fig:both}.
\begin{figure}[h]
\centering
\begin{subfigure}{.5\linewidth}
  \centering
  \includegraphics[width=1\linewidth]{lorenz.jpg}
  \caption{A subfigure}
  \label{fig:sub1}
\end{subfigure}%
\begin{subfigure}{.5\linewidth}
  \centering
  \includegraphics[width=1\linewidth]{bifurcation.png}
  \caption{A subfigure}
  \label{fig:sub2}
\end{subfigure}
\caption{A figure with two sub-figures}
\label{fig:both}
\end{figure}
Latex can ``optimize" the positioning of the figures, depending on what commands you give. The way this is done here is to let the compiler determine what the best position of the figure is. This means that you do not have direct control on where it ends up. Read more on this online.


\section{Lists and tables}
You can make lists like this
\begin{enumerate}
\item item 1
\item item 2
\begin{enumerate}
\item subitem
\end{enumerate}
\end{enumerate}
or like this
\begin{itemize}
\item item 1
\item item 2
\begin{itemize}
\item sub-item
\end{itemize}
\end{itemize}
Tables are bit more complicated, and there are a number of packages to be used. An example could look like this
\begin{table}[h]
\caption{A Table}\label{tab:table}
\begin{center}
\begin{tabular}{|l|c|c|c|}
\hline
 		& Col1 	& Col2 	& Col3 	\\[3pt] \hline
Row1	&  		&  		&  		\\[3pt] \hline
Row2&  		&  		&  		\\[3pt] \hline
Row3	&  		&  		&  		\\[3pt] \hline

\end{tabular}
\end{center}
\end{table}
and be referred to like this Table \ref{tab:table}


\section{Other}
If you like to input matlab snippets, you could do it like this
\begin{lstlisting}
r = 3.7;
x = 0.5;
for i = 1:N
	x = r*x*(1-x)
end
x = syms('x',[3,1],'real');
f = someComplicatedFucntion(x);
df = jacobian(f,x)';
string = 'str' % see, it also colors the comments and strings correctly
\end{lstlisting}
If you want to put in pseudo-code, you can do it like this
\begin{algorithm}
\caption{Newton}\label{alg:int}
\begin{algorithmic}[1]
\Procedure{Rootfinding}{$\epsilon$}
\State $x=\textsc{InitialGuess}()$
\While{$||f(x)|| < \epsilon $}
\State $[f(x), \nabla_x f(x) ] = \textsc{EvaluateFunction}(x)$
\State $\Delta x = \left(\nabla_x f(x) \right)^{-1}f(x)$
\State $\alpha = \textsc{GetStep}()$
\State $x = x-\alpha \Delta x$
\EndWhile
\Return{$x$}
\EndProcedure
\end{algorithmic}
\end{algorithm}
and refer to it like this Algorithm~\ref{alg:int}.
\end{document}
